\documentclass[11pt, a4paper]{article} % Dokumentklasse 'article', Schriftgröße 11pt, Papierformat A4

% --- Deutsche Spracheinstellungen und Zeichenkodierung ---
\usepackage[utf8]{inputenc}     % Erlaubt die direkte Eingabe von Umlauten etc. (UTF-8 Kodierung)
\usepackage[T1]{fontenc}        % Setzt die Schriftkodierung auf T1 (wichtig für Silbentrennung etc.)
\usepackage[ngerman]{babel}     % Lädt deutsche Spracheinstellungen (neue Rechtschreibung)

% --- Seitenlayout ---
\usepackage[a4paper, left=2.5cm, right=2.5cm, top=2.5cm, bottom=2.5cm]{geometry} % Seitenränder anpassen

% --- Nützliche Pakete ---
\usepackage{graphicx}           % Zum Einbinden von Grafiken (falls später doch benötigt)
\usepackage{hyperref}           % Erstellt klickbare Links (z.B. im Inhaltsverzeichnis)
\usepackage{amsmath}            % Für mathematische Formeln
\usepackage{amssymb}            % Für mathematische Symbole
\usepackage{listings}           % Zum Darstellen von Code-Snippets (falls benötigt)
%\usepackage{lipsum}             % Nur zum Erzeugen von Beispieltext (kann entfernt werden)
\usepackage{tocbibind}          % Fügt Inhaltsverzeichnis, Abbildungsverzeichnis etc. zum Inhaltsverzeichnis hinzu

% --- Dokumentinformationen (für PDF-Metadaten und ggf. Kopf-/Fußzeilen) ---
\title{Game Design Document: [Ihr Spielname]} % Titel für PDF Metadaten (Spielnamen anpassen)
\author{Team: [Ihr Teamname oder Namen der Mitglieder]} % Autor für PDF Metadaten (anpassen)

% --- Anpassungen ---
\hypersetup{                    % Einstellungen für hyperref
    colorlinks=true,            % Links farbig statt mit Rahmen
    linkcolor=black,            % Farbe für interne Links (z.B. Inhaltsverzeichnis) ist schwarz
    filecolor=magenta,          % Farbe für Dateilinks
    urlcolor=cyan,              % Farbe für URLs (kann auch auf black gesetzt werden)
    citecolor=black,            % Farbe für Zitat-Links (falls BibTeX genutzt wird)
    pdftitle={Game Design Document - [Ihr Spielname]}, % PDF-Titel anpassen (Spielnamen anpassen)
    pdfauthor={\@author},       % Nimmt den Autor aus \author für die PDF-Metadaten
    pdfpagemode=UseOutlines,    % Zeigt Lesezeichen beim Öffnen an (optional)
}

% Verhindert Einrückung am Absatzanfang global (optional, oft bei technischen Dokumenten gewünscht)
\setlength{\parindent}{0pt}
% Setzt einen kleinen Abstand zwischen Absätzen (wenn keine Einrückung verwendet wird)
\setlength{\parskip}{0.5em}


% ==================================================
% Beginn des eigentlichen Dokuments
% ==================================================
\begin{document}

% --- Benutzerdefinierte Titelseite ---
\begin{titlepage}
    \centering % Zentriert den Inhalt der Titelseite
    \vspace*{1cm} % Vertikaler Abstand vom oberen Rand (mit *)

    % --- Hochschule ---
    {\Large \textbf{Hochschule für angewandte Wissenschaften Coburg}}\par
    \vspace{0.5cm} % Vertikaler Abstand
    {\large Fakultät für Elektrotechnik und Informatik}\par
    \vspace{2.5cm} % Vertikaler Abstand

    % --- Dokumenttyp ---
    {\huge \bfseries Game Design Document}\par
    \vspace{1cm} % Vertikaler Abstand

    % --- Spieltitel ---
    {\HUGE \bfseries [Ihr Spielname]}\par
    \vspace{3cm} % Großer vertikaler Abstand

    % --- Autor(en) ---
    {\Large Vorgelegt von:}\par
    \vspace{0.5cm}
    {\large [Name Mitglied 1]}\par % Namen der Teammitglieder eintragen
    {\large [Name Mitglied 2]}\par
    {\large [Name Mitglied 3]}\par
    % {\large ...}\par % Fügen Sie weitere Mitglieder hinzu oder verwenden Sie einen Teamnamen
    % Alternativ: {\large Team: [Ihr Teamname]}
    \vspace{1.5cm}

    % --- Kurs/Modul/Betreuer ---
    {\Large Im Rahmen des Kurses:}\par
    \vspace{0.3cm}
    {\large Grundlagen des Game Design}\par % Kursname eingefügt
    \vspace{0.5cm}
    {\Large Betreuer:}\par
    \vspace{0.3cm}
    {\large Prof. Dr. Stephan Streuber}\par

    % --- Füllt den Rest der Seite nach unten ---
    \vfill

    % --- Datum ---
    {\large \today}\par % Aktuelles Datum (wird beim Kompilieren eingesetzt)
    % Alternativ ein festes Datum: {\large Bamberg, 1. April 2025} % (Datum anpassen)

\end{titlepage}
% --- Ende der Titelseite ---


\tableofcontents            % Erzeugt das Inhaltsverzeichnis
\newpage


% --- Abschnitt 1: Einführung ---
\section{Einführung}
\subsection{Spielname}
% Tragen Sie hier den vorläufigen oder endgültigen Namen des Spiels ein.
% Beispiel: Der Arbeitstitel des Projekts lautet "[Ihr Spielname]".
\subsection{Genre}
% Definieren Sie das Hauptgenre und ggf. Subgenres.
% Beispiel: 3D Puzzle-Platformer mit Fokus auf Physik-Rätseln.
\subsection{Plattformen}
% Listen Sie die Zielplattformen auf.
% Beispiel: PC (Windows, Steam).
\subsection{Zielgruppe}
% Beschreiben Sie die primäre und sekundäre Zielgruppe.
% Beispiel: SpielerInnen (12+) mit Interesse an Knobelspielen und entspannter Atmosphäre.
\subsection{USP (Unique Selling Proposition)}
% Was macht das Spiel einzigartig?
% Beispiel: Die Kernmechanik basiert auf der Manipulation von Zeit für einzelne Objekte in der Spielwelt.
\subsection{High Concept}
% Das Spielkonzept in einem Satz.
% Beispiel: "Ein atmosphärischer Puzzler, in dem Spieler durch Zeitmanipulation Hindernisse überwinden."

\newpage

% --- Abschnitt 2: Spielmechanik (Gameplay) ---
\section{Spielmechanik (Gameplay)}
\subsection{Kernmechaniken}
% Grundlegende Aktionen und Gameplay-Loop.
\subsection{Steuerung}
% Eingabemethoden und Schema.
\subsection{Spielmodi}
% Verfügbare Modi (Singleplayer, etc.).
\subsection{Ziel und Fortschritt}
% Spielziel, Gewinn-/Verlustbedingungen, Progression.
\subsection{Schwierigkeitskurve \& Balancing}
% Anstieg der Herausforderung, Anpassung.
\subsection{Wirtschaftssystem}
% Ressourcen, Währung, Belohnungen (falls zutreffend).

\newpage

% --- Abschnitt 3: Story & Charaktere ---
\section{Story \& Charaktere}
\subsection{Setting \& Lore}
% Weltbeschreibung, Hintergrundgeschichte.
\subsection{Hauptcharaktere}
% Protagonisten, Antagonisten, wichtige NPCs.
\subsection{Storystruktur}
% Handlungsverlauf, Narrative, Missionen/Quests.

\newpage

% --- Abschnitt 4: Spielwelt & Leveldesign ---
\section{Spielwelt \& Leveldesign}
\subsection{Umgebungen \& Karten}
% Zonen, Regionen, Levelübersicht.
\subsection{Levelstruktur}
% Linear, Open-World, etc., Designphilosophie.
\subsection{Interaktivität}
% Interaktive Elemente der Spielwelt.
\subsection{KI \& NPCs}
% Verhalten von Gegnern und anderen NPCs.

\newpage

% --- Abschnitt 5: Audiovisuelle Gestaltung ---
\section{Audiovisuelle Gestaltung}
\subsection{Grafikstil}
% Visueller Stil (Realistisch, Cartoon, etc.).
\subsection{Sounddesign \& Musik}
% Klanglandschaft, Musikstil, Effekte, Voice-Over.
\subsection{UI/UX Design}
% Menüs, HUD, Feedback, Benutzerfreundlichkeit.

\newpage

% --- Abschnitt 6: Technik & Entwicklung ---
\section{Technik \& Entwicklung}
\subsection{Game Engine}
% Genutzte Engine und Begründung.
\subsection{Programmiersprachen \& Tools}
% Sprachen, externe Software.
\subsection{Physik \& Animationen}
% Physik-Engine, Animationsmethoden.
\subsection{Netzwerk \& Multiplayer}
% Falls zutreffend: Architektur, Features.

\newpage

% --- Abschnitt 7: Projektmanagement & Teamstruktur ---
\section{Projektmanagement \& Teamstruktur}
\subsection{Entwicklungsteam}
% Rollenverteilung im Team. (Kann sich mit Titelseite überschneiden, hier ggf. detaillierter)
\subsection{Zeitplan \& Meilensteine}
% Grober Zeitplan, wichtige Deadlines/Phasen.
\subsection{Risiken \& Herausforderungen}
% Potenzielle Probleme und Lösungsansätze.

\end{document}
% ==================================================
% Ende des Dokuments
% ==================================================